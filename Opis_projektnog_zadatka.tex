\chapter{Opis projektnog zadatka}
		
		
		
		{Porastom broja neprofitnih organizacija iz godine u godinu, raste i potreba za kvalitetnim aplikacijama koje bi takvim organizacijama omogućile nesmetan rad i funkcioniranje. Najveći problem navedenih organizacija jest prihod. Kako nisu u mogućnosti samostalno postići zadovoljavajuću razinu prihoda, primorane su na suradnju s kompanijama sponzorima.}

		{Odaziv sponzora je često vrlo malen te je stoga potrebno kontaktirati velik broj kompanija. Praćenje statusa suradnji (uspješne / neuspješne) i osoba zaduženih za kontaktiranje sponzora postaje kompleksno i neefikasno, a također dolazi i do mogućnosti curenja informacija. Raste potreba za aplikacijom koja će omogućiti evidenciju navedenih stavki te korisnicima dodijeliti razinu ovlasti kako bi umanjili mogućnost curenja informacija.}

		{Cilj ovog projekta je razviti programsku podršku za stvaranje web aplikacije \textit{”Company Database“} koja će služiti za evidenciju statusa suradnje kompanija na projektima i jednostavan uvid od strane \underbar{ovlaštenih} korisnika.}\vspace{0.2cm}

		{Korisnici će se u web aplikaciju prijavljivati koristeći Google račun kojeg registriraju putem web aplikacije. Svaki je korisnik definiran sljedećim parametrima:}
		\begin{packed_item}
			\item {imenom}
			\item {prezimenom}
			\item {nadimkom}
			\item {kratkim opisom}
			\item {opseg projektnog zadatka}
			\item {najvišom razinom ovlasti koju posjeduje (ukoliko je pripadnik više različitih razina ovlasti)}
		\end{packed_item}

		{Svaki korisnik određene razine ovlasti ima sve mogućnosti svoje i svih nižih razina ovlasti sustava. Također ima mogućnost dodjeljivanja i oduzimanja \underbar{razina ovlasti} korisnicima nižih razina od svoje.}

		{Iznimka je \underbar{administrator}, najviša razina ovlasti sustava, koji može dodijeliti i oduzeti ovlasti svima, uključujući i ostale administratore sustava.}\\
\\
%vertical space za estetiku

		{Razlikovati ćemo 5 razina ovlasti, a to su redom od najviše prema najnižoj:}
		\begin{packed_item}
			\item {Administrator}
			\item {Moderator}
			\item {Fundraising (FR) responsible}
			\item {Fundraising (FR) team member}
			\item {Observer}
		\end{packed_item}

		{\underbar{Moderator} može dodati i obrisati kategoriju projekata, stvoriti i obrisati vlastite projekte te dodijeliti nekom korisniku razinu ovlasti FR responsible za vlastito stvoreni projekt.}\vspace{0.5}

		{\underbar{FR responsible} može za dodijeljeni projekt odabrati kompanije iz popisa kompanija s kojima želi surađivati na dodijeljenom projektu. Za svaku suradnju potrebno je odabrati korisnika koji će dobiti ovlasti FR team member koji u tom trenutku dobiva na email obavijest o zaduženju.}\vspace{0.5cm}

		{\underbar{FR team member} ima mogućnost:}
		\begin{packed_item}
			\item {modifikacije kategorije, statusa, komentara i vrijednosti suradnje}
			\item {pregleda i modificiranja svih podataka za dodijeljenu kompaniju}
		\end{packed_item}

		{Članovi FR tima određenog projekta dobivaju notifikaciju („ping”) na email i početni ekran na datum prvog i drugog „pinga“ tog projekta}			\vspace{0.5cm}

		{\underbar{Observer} ima mogućnost pregleda:}
		\begin{packed_item}
			\item {popisa svih projekata}
			\item {popisa svih kompanija (naziv i područje kompanije)}
			\item {informacija o projektu (datum početka, datum završetka, tko je FR responsible za navedeni projekt)}\vspace{0.3cm}
		\end{packed_item}

		{\underbar{Projekt} je obilježen nazivom, kategorijom, tipom, početkom i krajem (timestamp), FR responsible-om zaduženim za navedeni projekt, popisom FR team member-a, FR ciljem (željeni prihod od projekta), prvim i drugom „pingom“ (timestamp) te popisom kompanija.}\\
\\
\\
%vertical space za estetiku

		\vspace*{0.5}{\underbar{Kompanija} je definirana: nazivom, područjem kojim se bavi (npr. IT), mjesecom planiranja budžeta (bilo koji od 12 mjeseci u godini), državom, poštanskim brojem, gradom, adresom, linkom na web stranicu, kratkim opisom, boolean varijablom \textit{„kontaktirati“} (koja označava treba li tu tvrtku u budućnosti kontaktirati) te popisom zaposlenika.}\vspace{0.3cm}
		
		{\underbar{Popis kompanija} moguće je uzlazno i silazno sortirati prema:}
		\begin{packed_item}
			\item {nazivu}
			\item {području}
			\item {mjesecu planiranja budžeta}
			\item {linku na web stranicu}
		\end{packed_item}

		{Popis kompanija moguće je dodatno pretraživati po nazivu.}

		{\underbar{Zaposlenik} kao entitet pripada kompaniji, a definiran je imenom, prezimenom, email adresom, brojem mobitela, ulogom u kompaniji (npr. CEO) te kratkim opisom.}\vspace{0.3cm}

		{\underbar{Suradnja} predstavlja poveznicu između projekta i kompanije, a obilježena je FR team member-om koji je za nju zadužen, kontaktiranom osobom u kompaniji, kategorijom (financijska, materijalna ili akademska suradnja), vlastitim statusom (kontaktirano, ping, dopis, sastanak, uspješno ili neuspješno), komentarom / sažetkom suradnje te vrijednošću suradnje (koje se sumiraju i popunjavaju FR cilj – željeni prihod od projekta).}\\

		{Aplikacija, sa svim navedenim funkcionalnostima i specifičnostima, biti će izvedena kao web aplikacija kojoj korisnici pristupaju pomoću Google autentifikatora. Biti će jednostavna za korištenje zahvaljujući preglednom i intuitivnom sučelju.
Sustav će podržavati rad više korisnika u stvarnom vremenu, frontend će biti ostvaren u React-u, a backend će koristiti relacijsku bazu podataka i Spring boot.}

	
		\eject
		
	