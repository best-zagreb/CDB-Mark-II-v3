\chapter{Specifikacija programske potpore}
		
	\section{Funkcionalni zahtjevi}
			
			\textbf{\textit{dio 1. revizije}}\\
			
			\textit{Navesti \textbf{dionike} koji imaju \textbf{interes u ovom sustavu} ili  \textbf{su nositelji odgovornosti}. To su prije svega korisnici, ali i administratori sustava, naručitelji, razvojni tim.}\\
				
			\textit{Navesti \textbf{aktore} koji izravno \textbf{koriste} ili \textbf{komuniciraju sa sustavom}. Oni mogu imati inicijatorsku ulogu, tj. započinju određene procese u sustavu ili samo sudioničku ulogu, tj. obavljaju određeni posao. Za svakog aktora navesti funkcionalne zahtjeve koji se na njega odnose.}\\
			
			
			\noindent \textbf{Dionici:}
			
			\begin{packed_enum}
				
				\item Vlasnik neprofitabline organizacije
				\item Vlasnik kompanije
				\item Zaposlenik kompanije
					\item Kontakt
					\item Osoba zadužena za prodaju
				\item Zaposlenik neprofitabline organizacije
					\item Osoba zadužena za stvaranje projekata
					\item Organizator projekta
					\item Programska podrška
					\item Javni govornici
				\item Administrator
				\item Razvojni tim
				
			\end{packed_enum}
			
			\noindent \textbf{Aktori i njihovi funkcionalni zahtjevi:}
			
			
			\begin{packed_enum}
				\item  \underbar{ Neprijavljeni korisnik (inicijator) moze: }

				\begin{packed_enum}

					\item Ako je nadodan od strane administratora može se prijaviti svojim mailom
					\item U slučaju da je prvi neprijavljeni korisnik postaje administrator

				\end{packed_enum}

				\item  \underbar{Zaposlenik kompanije(Kontakt/Osoba zadužena za prodaju) (inicijator) može: }
				
				\begin{packed_enum}
					
					\item Pregledavati projekte
					\item Pregledavati kompanije (naziv i područje)
					\item Pregledavati informacija o projektima (datum početka, datum završetka, oganizator)
					\item Modifikacija svojih podataka
					
				\end{packed_enum}

				\item  \underbar{Zaposlenik neprofitabline organizacije(Programska podrška/Javni govornici) (inicijator) može: }

				\begin{packed_enum}

					\item Pregledavati projekte
					\item Pregledavati kompanije (naziv i područje)
					\item Pregledavati informacija o projektima (datum početka, datum završetka, FR-responsiblea)
					\item Modificirati kategorije projekata
					\item Modificirati u kojem je statusu suradnja
					\item Modificirati sažetake suradnja
					\item Modificirati vrijednosti suranje
					\item Pregledavati i modificirati sve podatke za dodijeljenu kompaniju
					\item Modifikacija svojih podataka

				\end{packed_enum}

				\item  \underbar{Zaposlenik neprofitabline organizacije(Organizator projekta) (inicijator) može: }

				\begin{packed_enum}

					\item Pregledavati projekte
					\item Pregledavati kompanije (naziv i područje)
					\item Pregledavati informacija o projektima (datum početka, datum završetka, FR-responsiblea)
					\item Modificirati kategorije projekata
					\item Modificirati u kojem je statusu suradnja
					\item Modificirati sažetake suradnja
					\item Modificirati vrijednosti suranje
					\item Pregledavati i modificirati sve podatke za dodijeljenu kompaniju
					\item Odabrati kompanije iz popisa kompanija s kojima će surađivati na projektu
					\item Modifikacija svojih podataka

				\end{packed_enum}

				\item  \underbar{ Osoba zadužena za stvaranje projekata (inicijator) može: }

				\begin{packed_enum}

					\item Pregledavati projekte
					\item Pregledavati kompanije (naziv i područje)
					\item Pregledavati informacija o projektima (datum početka, datum završetka, FR-responsiblea)
					\item Modificirati kategorije projekata
					\item Modificirati u kojem je statusu suradnja
					\item Modificirati sažetake suradnja
					\item Modificirati vrijednosti suranje
					\item Pregledavati i modificirati sve podatke za dodijeljenu kompaniju
					\item Odabrati kompanije iz popisa kompanija s kojima će surađivati na projektu
					\item Dodati i obrisati kategoriju projekata
					\item Stvoriti i obrisati vlastito stvorene projekte
					\item Dodijeliti nekom korisniku mogućnosti organizatora projekta na projektu koji je stvorio
					\item Modifikacija svojih podataka

				\end{packed_enum}

				\item  \underbar{ Administrator (inicijator) može: }

				\begin{packed_enum}

					\item Pregledavati projekte
					\item Pregledavati kompanije (naziv i područje)
					\item Pregledavati informacija o projektima (datum početka, datum završetka, FR-responsiblea)
					\item Modificirati kategorije projekata
					\item Modificirati u kojem je statusu suradnja
					\item Modificirati sažetake suradnja
					\item Modificirati vrijednosti suranje
					\item Pregledavati i modificirati sve podatke za dodijeljenu kompaniju
					\item Odabrati kompanije iz popisa kompanija s kojima će surađivati na projektu
					\item Dodati i obrisati kategoriju projekata
					\item Stvoriti i obrisati vlastito stvorene projekte
					\item Dodijeliti nekom korisniku bilo koju od prije navedenih pozicija
					\item Registrirati korisnika u sustav
					\item Registrirati sebe u sustav pri prvom pokretanju
					\item Modifikacija svojih podataka

				\end{packed_enum}
			
				\item  \underbar{Baza podataka (sudionik) može:}
				
				\begin{packed_enum}
					
					\item Pohranjuje sve podatke o projektima
					\item Pohranjuje sve podatke o kompanijama i njihovim kontaktima
					
				\end{packed_enum}

				\item  \underbar{Gmail api (sudionik) može:}

				\begin{packed_enum}

					\item Prijavljuje korisnika pomoću njegovog gmaila

				\end{packed_enum}
			\end{packed_enum}
			
			\eject 
			


			\subsection{Obrasci uporabe}
				
				\textbf{\textit{dio 1. revizije}}
				
				\subsubsection{Opis obrazaca uporabe}
					\textit{Funkcionalne zahtjeve razraditi u obliku obrazaca uporabe. Svaki obrazac je potrebno razraditi prema donjem predlošku. Ukoliko u nekom koraku može doći do odstupanja, potrebno je to odstupanje opisati i po mogućnosti ponuditi rješenje kojim bi se tijek obrasca vratio na osnovni tijek.}\\

					\noindent \underbar{\textbf{UC$<$1$>$ -$<$Registracija administratora$>$}}
					\begin{packed_item}

						\item \textbf{Glavni sudionik: }$<$Neprijavljeni korisnik$>$
						\item  \textbf{Cilj:} $<$Stvoriti administratov korisnički račun za pristup sustavu$>$
						\item  \textbf{Sudionici:} $<$Baza podataka, Gmail API$>$
						\item  \textbf{Preduvjet:} $<$-$>$
						\item  \textbf{Opis osnovnog tijeka:}

						\item[] \begin{packed_enum}

							\item $<$Korisnik dolazi na stranicu za prijavu$>$
							\item $<$Prebacuje se na googlovu prijavi gmailom$>$
							\item $<$Korisnik se prijavljuje mailom$>$
							\item $<$Dolazi na našu stranicu i dobiva rolu administratora$>$
						\end{packed_enum}
					\end{packed_item}

					\noindent \underbar{\textbf{UC$<$2$>$ -$<$Registracija$>$}}
					\begin{packed_item}

						\item \textbf{Glavni sudionik: }$<$Administrator$>$
						\item  \textbf{Cilj:} $<$Stvoriti korisnicki račun za pristup sustavu$>$
						\item  \textbf{Sudionici:} $<$Baza podataka$>$
						\item  \textbf{Preduvjet:} $<$-$>$
						\item  \textbf{Opis osnovnog tijeka:}

						\item[] \begin{packed_enum}

									\item $<$Administrator dolazi do svoje forme za registraciju$>$
									\item $<$Unosi gmail korisnika i njegovu rolu$>$
									\item $<$Dobiva obavijest o uspješnoj registraciji$>$
						\end{packed_enum}

						\item  \textbf{Opis mogućih odstupanja:}

						\item[] \begin{packed_item}

							\item[2.a] $<$Odabir vec zauzetog gmaila, unos korisničkih
							podatka u nedozvoljenom formatu ili pružanje neispravnoga gmaila$>$
							\item[] \begin{packed_enum}

								\item $<$Sustav obavještava administratora o neuspjelom upisu$>$
								\item $<$Administrator mijenja potrebne podatke te završava unos ili
								odustaje od registracije$>$

							\end{packed_enum}

						\end{packed_item}
					\end{packed_item}

					\noindent \underbar{\textbf{UC$<$3$>$ -$<$Prijava na stranicu$>$}}
					\begin{packed_item}
	
						\item \textbf{Glavni sudionik: }$<$Neprijavljeni korisnik$>$
						\item  \textbf{Cilj:} $<$Dati korisniku pristup stranici$>$
						\item  \textbf{Sudionici:} $<$Baza podataka$>$
						\item  \textbf{Preduvjet:} $<$Registracija$>$
						\item  \textbf{Opis osnovnog tijeka:}
						
						\item[] \begin{packed_enum}
	
							\item $<$Korisnik dolazi na stranicu za prijavu$>$
							\item $<$Prebacuje se na googlovu prijavi gmailom$>$
							\item $<$Korisnik se prijavljuje mailom$>$
							\item $<$Dolazi na našu stranicu$>$
						\end{packed_enum}
					\end{packed_item}

					\noindent \underbar{\textbf{UC$<$4$>$ -$<$Pregled projekata$>$}}
					\begin{packed_item}

						\item \textbf{Glavni sudionik: }$<$Zaposlenik kompanije$>$
						\item  \textbf{Cilj:} $<$Pregledati projekte$>$
						\item  \textbf{Sudionici:} $<$Baza podataka$>$
						\item  \textbf{Preduvjet:} $<$Klient je prijavljen$>$
						\item  \textbf{Opis osnovnog tijeka:}

						\item[] \begin{packed_enum}

							\item $<$Korisnik odabire opciiju "Projekti"$>$
							\item $<$Aplikacija prikazuje listu osnovnih podataka o projektima
							(naziv projekta, datum početka, datum završetka, organizatora)$>$
						\end{packed_enum}
					\end{packed_item}

					\noindent \underbar{\textbf{UC$<$5$>$ -$<$Pregled kompanija$>$}}
					\begin{packed_item}

						\item \textbf{Glavni sudionik: }$<$Zaposlenik kompanije$>$
						\item  \textbf{Cilj:} $<$Pregledati kompanije$>$
						\item  \textbf{Sudionici:} $<$Baza podataka$>$
						\item  \textbf{Preduvjet:} $<$Klient je prijavljen$>$
						\item  \textbf{Opis osnovnog tijeka:}

						\item[] \begin{packed_enum}

							\item $<$Korisnik odabire opciiju "Kompanije"$>$
							\item $<$Aplikacija prikazuje listu osnovnih podataka o kompanijama
							(naziv kompanije i područje)$>$
						\end{packed_enum}
					\end{packed_item}
					
				\subsubsection{Dijagrami obrazaca uporabe}
					
					\textit{Prikazati odnos aktora i obrazaca uporabe odgovarajućim UML dijagramom. Nije nužno nacrtati sve na jednom dijagramu. Modelirati po razinama apstrakcije i skupovima srodnih funkcionalnosti.}
				\eject		
				
			\subsection{Sekvencijski dijagrami}
				
				\textbf{\textit{dio 1. revizije}}\\
				
				\textit{Nacrtati sekvencijske dijagrame koji modeliraju najvažnije dijelove sustava (max. 4 dijagrama). Ukoliko postoji nedoumica oko odabira, razjasniti s asistentom. Uz svaki dijagram napisati detaljni opis dijagrama.}
				\eject
	
		\section{Ostali zahtjevi}
		
			\textbf{\textit{dio 1. revizije}}\\
		 
			 \textit{Nefunkcionalni zahtjevi i zahtjevi domene primjene dopunjuju funkcionalne zahtjeve. Oni opisuju \textbf{kako se sustav treba ponašati} i koja \textbf{ograničenja} treba poštivati (performanse, korisničko iskustvo, pouzdanost, standardi kvalitete, sigurnost...). Primjeri takvih zahtjeva u Vašem projektu mogu biti: podržani jezici korisničkog sučelja, vrijeme odziva, najveći mogući podržani broj korisnika, podržane web/mobilne platforme, razina zaštite (protokoli komunikacije, kriptiranje...)... Svaki takav zahtjev potrebno je navesti u jednoj ili dvije rečenice.}
			 
			 
			 
	