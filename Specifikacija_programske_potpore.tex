\chapter{Specifikacija programske potpore}
		
	\section{Funkcionalni zahtjevi}
			
			\textbf{\textit{dio 1. revizije}}\\
			
			\textit{Navesti \textbf{dionike} koji imaju \textbf{interes u ovom sustavu} ili  \textbf{su nositelji odgovornosti}. To su prije svega korisnici, ali i administratori sustava, naručitelji, razvojni tim.}\\
				
			\textit{Navesti \textbf{aktore} koji izravno \textbf{koriste} ili \textbf{komuniciraju sa sustavom}. Oni mogu imati inicijatorsku ulogu, tj. započinju određene procese u sustavu ili samo sudioničku ulogu, tj. obavljaju određeni posao. Za svakog aktora navesti funkcionalne zahtjeve koji se na njega odnose.}\\
			
			
			\noindent \textbf{Dionici:}
			
			\begin{packed_enum}
				
				\item Vlasnik neprofitabline organizacije
				\item Vlasnik kompanije
				\item Zaposlenik kompanije
					\item Kontakt
					\item Osoba zadužena za prodaju
				\item Zaposlenik neprofitabline organizacije
					\item Osoba zadužena za stvaranje projekata
					\item Organizator projekta
					\item Programska podrška
					\item Javni govornici
				\item Administrator
				\item Razvojni tim
				
			\end{packed_enum}
			
			\noindent \textbf{Aktori i njihovi funkcionalni zahtjevi:}
			
			
			\begin{packed_enum}
				\item  \underbar{ Neprijavljeni korisnik (inicijator) moze: }

				\begin{packed_enum}

					\item Ako je nadodan od strane administratora može se prijaviti svojim mailom
					\item U slučaju da je prvi neprijavljeni korisnik postaje administrator

				\end{packed_enum}

				\item  \underbar{Zaposlenik kompanije(Kontakt/Osoba zadužena za prodaju) (inicijator) može: }
				
				\begin{packed_enum}
					
					\item Pregledavati projekte
					\item Pregledavati kompanije (naziv i područje)
					\item Pregledavati informacija o projektima (datum početka, datum završetka, oganizator)
					\item Pregledavati suradnje
					\item Pregled kategorija projekata
					\item Promijeniti osobne podatake
					
				\end{packed_enum}

				\item  \underbar{Zaposlenik neprofitabline organizacije(Programska podrška/Javni govornici) (inicijator) može: }

				\begin{packed_enum}

					\item Pregledavati projekte
					\item Pregledavati kompanije (naziv i područje)
					\item Pregledavati informacija o projektima (datum početka, datum završetka, FR-responsiblea)
					\item Pregledavati suradnje
					\item Pregled kategorija projekata
					\item Promijeniti podatke o suradnji (kategorija, status, sažetak, vrijednost)
					\item Pregledavati i promijeniti sve podatke za dodijeljenu kompaniju
					\item Promijeniti osobne podatake

				\end{packed_enum}

				\item  \underbar{Zaposlenik neprofitabline organizacije(Organizator projekta) (inicijator) može: }

				\begin{packed_enum}

					\item Pregledavati projekte
					\item Pregledavati kompanije (naziv i područje)
					\item Pregledavati informacija o projektima (datum početka, datum završetka, FR-responsiblea)
					\item Pregledavati suradnje
					\item Pregled kategorija projekata
					\item Promijeniti podatke o suradnji (kategorija, status, sažetak, vrijednost)
					\item Pregledavati i promijeniti sve podatke za dodijeljenu kompaniju
					\item Strvoriti i obrisati suradnju
					\item Promijeniti osobne podatake

				\end{packed_enum}

				\item  \underbar{ Osoba zadužena za stvaranje projekata (inicijator) može: }

				\begin{packed_enum}

					\item Pregledavati projekte
					\item Pregledavati kompanije (naziv i područje)
					\item Pregledavati informacija o projektima (datum početka, datum završetka, FR-responsiblea)
					\item Pregledavati suradnje
					\item Pregled kategorija projekata
					\item Promijeniti podatke o suradnji (kategorija, status, sažetak, vrijednost)
					\item Pregledavati i promijeniti sve podatke za dodijeljenu kompaniju
					\item Strvoriti i obrisati suradnju
					\item Stvoriti i obrisati kategoriju projekata
					\item Stvoriti i obrisati vlastito stvorene projekte
					\item Dodijeliti i maknuti nekom korisniku mogućnosti organizatora projekta na projektu koji je stvorio
					\item Staviti i maknuti nekog korisnika s rolom zaposlenika neprofitabilne kompanije na projekt koji je stvorio
					\item Promijeniti osobne podatake

				\end{packed_enum}

				\item  \underbar{ Administrator (inicijator) može: }

				\begin{packed_enum}

					\item Pregledavati projekte
					\item Pregledavati kompanije (naziv i područje)
					\item Pregledavati informacija o projektima (datum početka, datum završetka, FR-responsiblea)
					\item Pregledavati suradnje
					\item Pregled kategorija projekata
					\item Promijeniti podatke o suradnji (kategorija, status, sažetak, vrijednost)
					\item Pregledavati i promijeniti sve podatke za dodijeljenu kompaniju
					\item Strvoriti i obrisati suradnju
					\item Stvoriti i obrisati kategoriju projekata
					\item Stvoriti i obrisati projekte
					\item Dodijeliti nekom korisniku mogućnosti organizatora projekta na projektu koji je stvorio
					\item Staviti nekog korisnika s rolom zaposlenika neprofitabilne kompanije na projekt koji je stvorio
					\item Pregledati korisnike
					\item Dodijeliti nekom korisniku bilo koju od prije navedenih pozicija
					\item Registrirati i maknuti korisnika iz sustava
					\item Registrirati sebe u sustav pri prvom pokretanju
					\item Promijeniti osobne podatake

				\end{packed_enum}
			
				\item  \underbar{Baza podataka (sudionik) može:}
				
				\begin{packed_enum}
					
					\item Pohranjuje sve podatke o projektima
					\item Pohranjuje sve podatke o kompanijama i njihovim kontaktima
					
				\end{packed_enum}

				\item  \underbar{Gmail api (sudionik) može:}

				\begin{packed_enum}

					\item Prijavljuje korisnika pomoću njegovog gmaila

				\end{packed_enum}
			\end{packed_enum}
			
			\eject 
			


			\subsection{Obrasci uporabe}
				
				\textbf{\textit{dio 1. revizije}}
				
				\subsubsection{Opis obrazaca uporabe}
					\textit{Funkcionalne zahtjeve razraditi u obliku obrazaca uporabe. Svaki obrazac je potrebno razraditi prema donjem predlošku. Ukoliko u nekom koraku može doći do odstupanja, potrebno je to odstupanje opisati i po mogućnosti ponuditi rješenje kojim bi se tijek obrasca vratio na osnovni tijek.}\\

					\noindent \underbar{\textbf{UC$<$1$>$ -$<$Registracija administratora$>$}}
					\begin{packed_item}

						\item \textbf{Glavni sudionik: }$<$Neprijavljeni korisnik$>$
						\item  \textbf{Cilj:} $<$Stvoriti administratov korisnički račun za pristup sustavu$>$
						\item  \textbf{Sudionici:} $<$Baza podataka, Gmail API$>$
						\item  \textbf{Preduvjet:} $<$-$>$
						\item  \textbf{Opis osnovnog tijeka:}

						\item[] \begin{packed_enum}

							\item $<$Korisnik dolazi na stranicu za prijavu$>$
							\item $<$Prebacuje se na googlovu prijavi gmailom$>$
							\item $<$Korisnik se prijavljuje mailom$>$
							\item $<$Dolazi na našu stranicu i dobiva rolu administratora$>$
						\end{packed_enum}
					\end{packed_item}

					\noindent \underbar{\textbf{UC$<$2$>$ -$<$Registracija$>$}}
					\begin{packed_item}

						\item \textbf{Glavni sudionik: }$<$Administrator$>$
						\item  \textbf{Cilj:} $<$Stvoriti korisnicki račun za pristup sustavu$>$
						\item  \textbf{Sudionici:} $<$Baza podataka$>$
						\item  \textbf{Preduvjet:} $<$-$>$
						\item  \textbf{Opis osnovnog tijeka:}

						\item[] \begin{packed_enum}

							\item $<$Korisnik odabire opciju "Korisnici"$>$
							\item $<$Aplikacija prikazuje listu korisnika$>$
							\item $<$Korisnik odabire opciju za stvaranje korisnika$>$
							\item $<$Aplikacija prikaže formu za stvaranje korisnika$>$
							\item $<$Korisnik popuni potrebne podatke(email i rolu)$>$
							\item $<$Korisnik odabere opciju spremi$>$
							\item $<$Baza podataka se ažurira$>$
							\item $<$Aplikacija vraća korisnika na novu listu korisnika$>$
						\end{packed_enum}

						\item  \textbf{Opis mogućih odstupanja:}

						\item[] \begin{packed_item}

							\item[2.a] $<$Odabir vec zauzetog gmaila, unos korisničkih
							podatka u nedozvoljenom formatu ili pružanje neispravnoga gmaila$>$
							\item[] \begin{packed_enum}

								\item $<$Sustav obavještava administratora o neuspjelom upisu$>$
								\item $<$Administrator mijenja potrebne podatke te završava unos ili
								odustaje od registracije$>$

							\end{packed_enum}

						\end{packed_item}
					\end{packed_item}

					\noindent \underbar{\textbf{UC$<$3$>$ -$<$Prijava na stranicu$>$}}
					\begin{packed_item}
	
						\item \textbf{Glavni sudionik: }$<$Neprijavljeni korisnik$>$
						\item  \textbf{Cilj:} $<$Dati korisniku pristup stranici$>$
						\item  \textbf{Sudionici:} $<$Baza podataka$>$
						\item  \textbf{Preduvjet:} $<$Registracija$>$
						\item  \textbf{Opis osnovnog tijeka:}
						
						\item[] \begin{packed_enum}
	
							\item $<$Korisnik dolazi na stranicu za prijavu$>$
							\item $<$Prebacuje se na googlovu prijavi gmailom$>$
							\item $<$Korisnik se prijavljuje mailom$>$
							\item $<$Dolazi na našu stranicu$>$
						\end{packed_enum}
					\end{packed_item}

					\noindent \underbar{\textbf{UC$<$4$>$ -$<$Pregled kategorija projekata$>$}}
					\begin{packed_item}

						\item \textbf{Glavni sudionik: }$<$Zaposlenik kompanije$>$
						\item  \textbf{Cilj:} $<$Pregledati kategorije projekta$>$
						\item  \textbf{Sudionici:} $<$Baza podataka$>$
						\item  \textbf{Preduvjet:} $<$Korisnik je prijavljen$>$
						\item  \textbf{Opis osnovnog tijeka:}

						\item[] \begin{packed_enum}

							\item $<$Korisnik odabire opciju "Projekti"$>$
							\item $<$Aplikacija prikazuje listu kategorija projekata$>$
						\end{packed_enum}
					\end{packed_item}

					\noindent \underbar{\textbf{UC$<$5$>$ -$<$Pregled projekata$>$}}
					\begin{packed_item}

						\item \textbf{Glavni sudionik: }$<$Zaposlenik kompanije$>$
						\item  \textbf{Cilj:} $<$Pregledati projekte$>$
						\item  \textbf{Sudionici:} $<$Baza podataka$>$
						\item  \textbf{Preduvjet:} $<$Korisnik je prijavljen$>$
						\item  \textbf{Opis osnovnog tijeka:}

						\item[] \begin{packed_enum}

							\item $<$Korisnik odabire opciju "Projekti"$>$
							\item $<$Aplikacija prikazuje listu kategorija projekata$>$
							\item $<$Korisnik odabire kategoriiju projekta$>$
							\item $<$Aplikacija prikazuje listu projekata (samo imena) te kategorije$>$
						\end{packed_enum}
					\end{packed_item}

					\noindent \underbar{\textbf{UC$<$6$>$ -$<$Pregled kompanija$>$}}
					\begin{packed_item}

						\item \textbf{Glavni sudionik: }$<$Zaposlenik kompanije$>$
						\item  \textbf{Cilj:} $<$Pregledati kompanije$>$
						\item  \textbf{Sudionici:} $<$Baza podataka$>$
						\item  \textbf{Preduvjet:} $<$Klient je prijavljen$>$
						\item  \textbf{Opis osnovnog tijeka:}

						\item[] \begin{packed_enum}

							\item $<$Korisnik odabire opciju "Kompanije"$>$
							\item $<$Aplikacija prikazuje listu osnovnih podataka o kompanijama
							(naziv kompanije i područje)$>$
						\end{packed_enum}
					\end{packed_item}

					\noindent \underbar{\textbf{UC$<$7$>$ -$<$Pregled informacija o projektu$>$}}
					\begin{packed_item}

						\item \textbf{Glavni sudionik: }$<$Zaposlenik kompanije$>$
						\item  \textbf{Cilj:} $<$Pregledati informacije o projektu$>$
						\item  \textbf{Sudionici:} $<$Baza podataka$>$
						\item  \textbf{Preduvjet:} $<$Korisnik je prijavljen$>$
						\item  \textbf{Opis osnovnog tijeka:}

						\item[] \begin{packed_enum}

							\item $<$Korisnik odabire opciju "Projekti"$>$
							\item $<$Aplikacija prikazuje listu kategorija projekata$>$
							\item $<$Korisnik odabire kategoriiju projekta$>$
							\item $<$Aplikacija prikazuje listu projekata (samo imena) te kategorije$>$
							\item $<$Korisnik klikne na ime projekta o kojem želi informacije$>$
							\item $<$Prikažu se detalji tog projekta (naziv projekta, datum početka,
							datum završetka, organizator) na lijevoj strani ekrana$>$
						\end{packed_enum}
					\end{packed_item}

					\noindent \underbar{\textbf{UC$<$8$>$ -$<$Pregled osobnih podataka$>$}}
					\begin{packed_item}

						\item \textbf{Glavni sudionik: }$<$Zaposlenik kompanije$>$
						\item  \textbf{Cilj:} $<$Pregledati osobne podatke$>$
						\item  \textbf{Sudionici:} $<$Baza podataka$>$
						\item  \textbf{Preduvjet:} $<$Korisnik je prijavljen$>$
						\item  \textbf{Opis osnovnog tijeka:}

						\item[] \begin{packed_enum}

							\item $<$Korisnik odabire opciju "Osobni podatci"$>$
							\item $<$Aplikacija prikazuje osobne podatke korisnika$>$
						\end{packed_enum}
					\end{packed_item}

					\noindent \underbar{\textbf{UC$<$9$>$ -$<$Promjena osobnih podataka$>$}}
					\begin{packed_item}

						\item \textbf{Glavni sudionik: }$<$Zaposlenik kompanije$>$
						\item  \textbf{Cilj:} $<$Promijeniti osobne podatke$>$
						\item  \textbf{Sudionici:} $<$Baza podataka$>$
						\item  \textbf{Preduvjet:} $<$Korisnik je prijavljen$>$
						\item  \textbf{Opis osnovnog tijeka:}

						\item[] \begin{packed_enum}

							\item $<$Korisnik odabire opciju "Osobni podatci"$>$
							\item $<$Korisnik odabere opciju "Promijeni osobne podatke"$>$
							\item $<$Korisnik mijenja svoje osobne podatke$>$
							\item $<$Korisnik sprema promjene$>$
							\item $<$Baza podataka se ažurira$>$
							\item $<$Aplikacija vraća korisnika na početnu stranicu$>$
						\end{packed_enum}

						\item  \textbf{Opis mogućih odstupanja:}

						\item[] \begin{packed_item}

							\item[3.a] $<$Korisnik promijeni svoje osobne podatke,
							ali ne odabere opciju "Spremi"$>$
							\item[] \begin{packed_enum}

								\item $<$Sustav obavještava korisnika da nije spremio podatke prije izlaska
								iz prozora i nudi mu opcije da spremi i izađe ili samo izađe$>$

							\end{packed_enum}

						\end{packed_item}
					\end{packed_item}

					\noindent \underbar{\textbf{UC$<$10$>$ -$<$Pregled suradnji$>$}}
					\begin{packed_item}
					
						\item \textbf{Glavni sudionik: }$<$Zaposlenik kompanije$>$
						\item  \textbf{Cilj:} $<$Promijeniti podatke o suradnji$>$
						\item  \textbf{Sudionici:} $<$Baza podataka$>$
						\item  \textbf{Preduvjet:} $<$Korisnik je prijavljen$>$
						\item  \textbf{Opis osnovnog tijeka:}
					
						\item[] \begin{packed_enum}

							\item $<$Prvi način$>$

							\item[] \begin{packed_item}

								\item $<$Korisnik odabire opciju "Projekti"$>$
								\item $<$Aplikacija prikazuje listu kategorija projekata$>$
								\item $<$Korisnik odabire kategoriiju projekta$>$
								\item $<$Aplikacija prikazuje listu projekata (samo imena) te kategorije$>$
								\item $<$Korisnik klikne na ime projekta u kojem se nalazi suradnja koja ga zanima$>$
								\item $<$Aplikacija prikazuje tablicu suradnji vezanih uz taj projekt (ime kompanije,
								organizator, status, kontakt, sažetak, akcije) na desnoj strani ekrana$>$

							\end{packed_item}

							\item $<$Drugi način$>$

							\item[] \begin{packed_item}

								\item $<$Korisnik odabire opciju "Kompanije"$>$
								\item $<$Aplikacija prikazuje listu kompanija (naziv i područje)$>$
								\item $<$Korisnik klikne na gumb za detalje kompanije u kojem se nalazi suradnja koja ga zanima$>$
								\item $<$Aplikacija prikazuje tablicu suradnji vezanih uz tu kompaniju (ime projekta,
								organizator, status, kontakt, sažetak, akcije) na desnoj strani ekrana$>$

							\end{packed_item}

						\end{packed_enum}
					\end{packed_item}

					\noindent \underbar{\textbf{UC$<$11$>$ -$<$Promjena podataka o suradnji$>$}}
					\begin{packed_item}

						\item \textbf{Glavni sudionik: }$<$Zaposlenik neprofitabline organizacije$>$
						\item  \textbf{Cilj:} $<$Promijeniti podatke o suradnji$>$
						\item  \textbf{Sudionici:} $<$Baza podataka$>$
						\item  \textbf{Preduvjet:} $<$Korisnik je prijavljen$>$
						\item  \textbf{Opis osnovnog tijeka:}

						\item[] \begin{packed_enum}

							\item $<$Prvi način$>$

							\item[] \begin{packed_item}

								\item $<$Korisnik odabire opciju "Projekti"$>$
								\item $<$Aplikacija prikazuje listu kategorija projekata$>$
								\item $<$Korisnik odabire kategoriiju projekta$>$
								\item $<$Aplikacija prikazuje listu projekata (samo imena) te kategorije$>$
								\item $<$Korisnik klikne na ime projekta u kojem se nalazi suradnja koja ga zanima$>$
								\item $<$Aplikacija prikazuje tablicu suradnji (ime kompanije, organizator, status,
								kontakt, sažetak, akcije) na desnoj strani ekrana$>$
								\item $<$Korisnik klikne na gumb "uredi" u najdesnijem polju tablice suradnji$>$
								\item $<$Aplikacija prikaže stranicu samo s podatcima te suradnje$>$
								\item $<$Korisnik mijenja podatke o suradnji$>$
								\item $<$Korisnik sprema promjene$>$
								\item $<$Baza podataka se ažurira$>$
								\item $<$Aplikacija vraća korisnika na novu tablicu suradnji$>$

							\end{packed_item}

							\item $<$Drugi način$>$

							\item[] \begin{packed_item}

								\item $<$Korisnik odabire opciju "Kompanije"$>$
								\item $<$Aplikacija prikazuje listu kompanija (naziv i područje)$>$
								\item $<$Korisnik klikne na gumb za detalje kompanije u kojem se nalazi suradnja koja ga zanima$>$
								\item $<$Aplikacija prikazuje tablicu suradnji vezanih uz tu kompaniju (ime projekta,
								žaduženi, status, kontakt, sažetak, akcije) na desnoj strani ekrana$>$
								\item $<$Korisnik klikne na gumb "uredi" u najdesnijem polju tablice suradnji$>$
								\item $<$Aplikacija prikaže stranicu samo s podatcima te suradnje$>$
								\item $<$Korisnik mijenja podatke o suradnji$>$
								\item $<$Korisnik sprema promjene$>$
								\item $<$Baza podataka se ažurira$>$
								\item $<$Aplikacija vraća korisnika na novu tablicu suradnji$>$

							\end{packed_item}

							\item  \textbf{Opis mogućih odstupanja:}

							\item[] \begin{packed_item}

								\item[1.9.a, 2.7.a] $<$Korisnik promijeni podatke o suradnji,
								ali ne odabere opciju "Spremi"$>$
								\item[] \begin{packed_enum}

									\item $<$Sustav obavještava korisnika da nije spremio podatke prije izlaska
									iz prozora i nudi mu opcije da spremi i izađe ili samo izađe$>$

								\end{packed_enum}

								\item[1.9.b, 2.7.b] $<$Korisnik promijeni podatke o suradnji na nevažeće vrijednositi
								i odabere opciju "Spremi"$>$
								\item[] \begin{packed_enum}

									\item $<$Sustav obavještava korisnika da nove vrijednosti nisu valjanje$>$

								\end{packed_enum}

							\end{packed_item}

						\end{packed_enum}
					\end{packed_item}

					\noindent \underbar{\textbf{UC$<$12$>$ -$<$Pregled svih podataka o kompaniji$>$}}
					\begin{packed_item}

						\item \textbf{Glavni sudionik: }$<$Zaposlenik neprofitabline organizacije$>$
						\item  \textbf{Cilj:} $<$Pregledati sve podate o kompaniji$>$
						\item  \textbf{Sudionici:} $<$Baza podataka$>$
						\item  \textbf{Preduvjet:} $<$Korisnik je prijavljen$>$
						\item  \textbf{Opis osnovnog tijeka:}

						\item[] \begin{packed_enum}

							\item $<$Korisnik odabire jednu od opcija "Kompanije"$>$
							\item $<$Aplikacija prikaže listu kompanija$>$
							\item $<$Korisnik klikne na gumb za detalje pored kompanije na koju je dodjeljen$>$
							\item $<$Aplikacija prikaže detalje o toj kompaniji na lijevoj strani ekrana$>$
						\end{packed_enum}
					\end{packed_item}

					\noindent \underbar{\textbf{UC$<$13$>$ -$<$Promjena podataka o kompaniji$>$}}
					\begin{packed_item}

						\item \textbf{Glavni sudionik: }$<$Zaposlenik neprofitabline organizacije$>$
						\item  \textbf{Cilj:} $<$Promijeniti podate o kompaniji$>$
						\item  \textbf{Sudionici:} $<$Baza podataka$>$
						\item  \textbf{Preduvjet:} $<$Korisnik je prijavljen$>$
						\item  \textbf{Opis osnovnog tijeka:}

						\item[] \begin{packed_enum}

							\item $<$Korisnik odabire jednu od opcija "Kompanije"$>$
							\item $<$Aplikacija prikaže listu kompanija$>$
							\item $<$Korisnik klikne na gumb za detalje pored kompanije na koju je dodjeljen$>$
							\item $<$Aplikacija prikaže detalje o toj kompaniji na lijevoj strani ekrana$>$
							\item $<$Korisnik klikne gumb za editiranje koji se nalazi taman iznad detalja o kompaniji$>$
							\item $<$Korisnik mijenja podatke o kompaniji$>$
							\item $<$Korisnik sprema promjene$>$
							\item $<$Baza podataka se ažurira$>$
							\item $<$Aplikacija vraća korisnika na novu listu kompanija$>$
						\end{packed_enum}

						\item  \textbf{Opis mogućih odstupanja:}

						\item[] \begin{packed_item}

							\item[5.a] $<$Korisnik promijeni podatke o kompaniji, ali ne odabere opciju "Spremi"$>$
							\item[] \begin{packed_enum}

								\item $<$Sustav obavještava korisnika da nije spremio podatke prije izlaska
								iz prozora i nudi mu opcije da spremi i izađe ili samo izađe$>$

							\end{packed_enum}

							\item[5.b] $<$Korisnik promijeni podatke o kompaniji na nevaljane podatke i odabere opciju "Spremi"$>$
							\item[] \begin{packed_enum}

								\item $<$Sustav obavještava korisnika da su podatci nevaljani$>$

							\end{packed_enum}

						\end{packed_item}
					\end{packed_item}

					\noindent \underbar{\textbf{UC$<$14$>$ -$<$Stvaranje suradnje za dodijeljen projekt $>$}}
					\begin{packed_item}

						\item \textbf{Glavni sudionik: }$<$Organizator$>$
						\item  \textbf{Cilj:} $<$Stvoriti suradnju za dodijeljen projekt$>$
						\item  \textbf{Sudionici:} $<$Baza podataka$>$
						\item  \textbf{Preduvjet:} $<$Korisnik je prijavljen$>$
						\item  \textbf{Opis osnovnog tijeka:}

						\item[] \begin{packed_enum}

							\item $<$Korisnik odabire opciju "Projekti"$>$
							\item $<$Aplikacija prikazuje listu kategorija projekata$>$
							\item $<$Korisnik odabire kategoriiju projekta$>$
							\item $<$Aplikacija prikazuje listu projekata (samo imena) te kategorije$>$
							\item $<$Korisnik klikne na ime projekta u kojem se nalazi suradnja koja ga zanima$>$
							\item $<$Aplikacija prikazuje tablicu suradnji (ime kompanije, organizator, status,
							kontakt, sažetak, akcije) na desnoj strani ekrana$>$
							\item $<$Korisnik odabere opciju "Nova suradnja"$>$
							\item $<$Aplikacija prikaže formu za stvaranje nove suradnje$>$
							\item $<$Korisnik odabere kompaniju i upiše podatke o suradnji$>$
							\item $<$Korisnik sprema promjene$>$
							\item $<$Baza podataka se ažurira$>$
							\item $<$Aplikacija vraća korisnika na novu listu suradnji$>$
						\end{packed_enum}

						\item  \textbf{Opis mogućih odstupanja:}

						\item[] \begin{packed_item}

							\item[8.a] $<$Korisnik upiše podatke o suradnji, ali ne odabere opciju "Spremi"$>$
							\item[] \begin{packed_enum}

								\item $<$Sustav obavještava korisnika da nije spremio podatke prije izlaska
								iz prozora i nudi mu opcije da spremi i izađe ili samo izađe$>$

							\end{packed_enum}

							\item[8.b] $<$Korisnik odabere opciju "Spremi", ali ne upiše sve nužne podatke za suradnju$>$
							\item[] \begin{packed_enum}

								\item $<$Sustav obavještava korisnika da nije svaki nužan podatak upisan$>$

							\end{packed_enum}

							\item[8.c] $<$Korisnik odabere opciju "Spremi", ali neki od podataka je neispravan$>$
							\item[] \begin{packed_enum}

								\item $<$Sustav obavještava korisnika za neispravan podatak$>$

							\end{packed_enum}

						\end{packed_item}
					\end{packed_item}

					\noindent \underbar{\textbf{UC$<$15$>$ -$<$Micanje suradnje sa dodijeljenog projekta$>$}}
					\begin{packed_item}

						\item \textbf{Glavni sudionik: }$<$Organizator$>$
						\item  \textbf{Cilj:} $<$Maknuti suradnju sa dodijeljenog projekta$>$
						\item  \textbf{Sudionici:} $<$Baza podataka$>$
						\item  \textbf{Preduvjet:} $<$Korisnik je prijavljen$>$
						\item  \textbf{Opis osnovnog tijeka:}

						\item[] \begin{packed_enum}

									\item $<$Korisnik odabire opciju "Projekti"$>$
									\item $<$Aplikacija prikazuje listu kategorija projekata$>$
									\item $<$Korisnik odabire kategoriiju projekta$>$
									\item $<$Aplikacija prikazuje listu projekata (samo imena) te kategorije$>$
									\item $<$Korisnik klikne na ime projekta u kojem se nalazi suradnja koja ga zanima$>$
									\item $<$Aplikacija prikazuje tablicu suradnji (ime kompanije, organizator, status,
									kontakt, sažetak, akcije) na desnoj strani ekrana$>$
									\item $<$Korisnik odabere opciju za micanje suradnje koja se nalazi pored same suradnje$>$
									\item $<$Aplikacija pita korisnika je li uvijeren da želi maknuti suradnju$>$
									\item $<$Korisnik odabere opciju da želi$>$
									\item $<$Baza podataka se ažurira$>$
									\item $<$Aplikacija vraća korisnika na novu listu suradnji$>$
						\end{packed_enum}
					\end{packed_item}

					\noindent \underbar{\textbf{UC$<$16$>$ -$<$Dodavanje kategorije projekta$>$}}
					\begin{packed_item}

						\item \textbf{Glavni sudionik: }$<$Osoba zadužena za stvaranje projekata$>$
						\item  \textbf{Cilj:} $<$Dodati kategoriju projekta$>$
						\item  \textbf{Sudionici:} $<$Baza podataka$>$
						\item  \textbf{Preduvjet:} $<$Korisnik je prijavljen$>$
						\item  \textbf{Opis osnovnog tijeka:}

						\item[] \begin{packed_enum}

							\item $<$Korisnik odabire opciju "Projekti"$>$
							\item $<$Aplikacija prikazuje listu kategorija projekata$>$
							\item $<$Korisnik odabire opciju "Nova kategorija"$>$
							\item $<$Aplikacija prikazuje formu za stvaranje nove kategorije projekta$>$
							\item $<$Korisnik upiše ime nove kategorije projekta$>$
							\item $<$Korisnik sprema promjene$>$
							\item $<$Baza podataka se ažurira$>$
							\item $<$Aplikacija vraća korisnika na novu listu kategorij projekata$>$
						\end{packed_enum}

						\item  \textbf{Opis mogućih odstupanja:}

						\item[] \begin{packed_item}

							\item[5.a] $<$Korisnik upiše ime nove kategorije projekta, ali ne odabere opciju "Spremi"$>$
							\item[] \begin{packed_enum}

								\item $<$Sustav obavještava korisnika da nije spremio podatke prije izlaska
								iz prozora i nudi mu opcije da spremi i izađe ili samo izađe$>$

							\end{packed_enum}

							\item[5.b] $<$Korisnik odabere opciju "Spremi", ali ne upiše ime za kategoriju projekta$>$
							\item[] \begin{packed_enum}

								\item $<$Sustav obavještava korisnika da nije upisao ime$>$

							\end{packed_enum}

						\end{packed_item}
					\end{packed_item}

					\noindent \underbar{\textbf{UC$<$17$>$ -$<$Brisanje kategorije projekta$>$}}
					\begin{packed_item}

						\item \textbf{Glavni sudionik: }$<$Osoba zadužena za stvaranje projekata$>$
						\item  \textbf{Cilj:} $<$Izbrisati kategoriju projekta$>$
						\item  \textbf{Sudionici:} $<$Baza podataka$>$
						\item  \textbf{Preduvjet:} $<$Korisnik je prijavljen$>$
						\item  \textbf{Opis osnovnog tijeka:}

						\item[] \begin{packed_enum}

							\item $<$Korisnik odabire opciju "Projekti"$>$
							\item $<$Aplikacija prikazuje listu kategorija projekata$>$
							\item $<$Korisnik odabire opciju za brisanje pored kategorije$>$
							\item $<$Aplikacija prikaže popup koji pita korisnika je li siguran$>$
							\item $<$Korisnik odabere opciju da je siguran$>$
							\item $<$Baza podataka se ažurira$>$
							\item $<$Aplikacija vraća korisnika na novu listu kategorija projekata$>$
						\end{packed_enum}

						\item  \textbf{Opis mogućih odstupanja:}

						\item[] \begin{packed_item}

							\item[5.a] $<$Kategorija koju korisnik želi izbrisati ima projekte u sebi$>$
							\item[] \begin{packed_enum}

								\item $<$Sustav obavještava korisnika da postoje projekti te kategorije i obavještava ga
								da prije nego što može izbrisati kategoriju mora maknuti te projekte iz nje$>$

							\end{packed_enum}

						\end{packed_item}
					\end{packed_item}

					\noindent \underbar{\textbf{UC$<$18$>$ -$<$Stvaranje projekta$>$}}
					\begin{packed_item}

						\item \textbf{Glavni sudionik: }$<$Osoba zadužena za stvaranje projekata$>$
						\item  \textbf{Cilj:} $<$Stvoriti projekt$>$
						\item  \textbf{Sudionici:} $<$Baza podataka$>$
						\item  \textbf{Preduvjet:} $<$Korisnik je prijavljen$>$
						\item  \textbf{Opis osnovnog tijeka:}

						\item[] \begin{packed_enum}

							\item $<$Korisnik odabire opciju "Projekti"$>$
							\item $<$Aplikacija prikazuje listu kategorija projekata$>$
							\item $<$Korisnik odabire kategoriiju projekta$>$
							\item $<$Aplikacija prikazuje listu projekata (samo imena) te kategorije$>$
							\item $<$Korisnik odabire opciju "Novi projekti"$>$
							\item $<$Aplikacija prikazuje formu za stvaranje novog projekta$>$
							\item $<$Korisnik popuni podatke za stvaranje novog projekta$>$
							\item $<$Korisnik sprema promjene$>$
							\item $<$Baza podataka se ažurira$>$
							\item $<$Aplikacija vraća korisnika na novu listu projekata$>$
						\end{packed_enum}

						\item  \textbf{Opis mogućih odstupanja:}

						\item[] \begin{packed_item}

							\item[7.a] $<$Korisnik upiše podatke za novi projekt, ali ne odabere opciju "Spremi"$>$
							\item[] \begin{packed_enum}

								\item $<$Sustav obavještava korisnika da nije spremio podatke prije izlaska
								iz prozora i nudi mu opcije da spremi i izađe ili samo izađe$>$

							\end{packed_enum}

							\item[7.b] $<$Korisnik odabere opciju "Spremi", ali ne upiše neke nužne podatke za 
							stvaranje novog projekta$>$
							\item[] \begin{packed_enum}

								\item $<$Sustav obavještava korisnika da nije popunio sve nužne podatke za 
								stvaranje novog projekta$>$

							\end{packed_enum}

						\end{packed_item}
					\end{packed_item}

					\noindent \underbar{\textbf{UC$<$19$>$ -$<$Brisanje projekta$>$}}
					\begin{packed_item}
					
						\item \textbf{Glavni sudionik: }$<$Osoba zadužena za stvaranje projekata$>$
						\item  \textbf{Cilj:} $<$Izbrisati projekt$>$
						\item  \textbf{Sudionici:} $<$Baza podataka$>$
						\item  \textbf{Preduvjet:} $<$Korisnik je prijavljen$>$
						\item  \textbf{Opis osnovnog tijeka:}
					
						\item[] \begin{packed_enum}

							\item $<$Korisnik odabire opciju "Projekti"$>$
							\item $<$Aplikacija prikazuje listu kategorija projekata$>$
							\item $<$Korisnik odabire kategoriiju projekta$>$
							\item $<$Aplikacija prikazuje listu projekata (samo imena) te kategorije$>$
							\item $<$Korisnik odabire opciju za brisanje pored projekta$>$
							\item $<$Aplikacija prikaže popup koji pita korisnika je li siguran$>$
							\item $<$Korisnik odabere opciju da je siguran$>$
							\item $<$Baza podataka se ažurira$>$
							\item $<$Aplikacija vraća korisnika na novu listu projekata$>$
						\end{packed_enum}
					
						\item  \textbf{Opis mogućih odstupanja:}
					
						\item[] \begin{packed_item}
					
							\item[5.a] $<$Korisnik nije stvorio ovaj projekt$>$
							\item[] \begin{packed_enum}
			
								\item $<$Sustav obavještava korisnika da nije stvorio taj pojekt, pa ga ne može izbrisati.
								Ovo se ne pojavljuje ako je korisnik administrator.$>$
			
							\end{packed_enum}
					
						\end{packed_item}
					\end{packed_item}

					\noindent \underbar{\textbf{UC$<$20$>$ -$<$Dodavanje organizatora na projekt$>$}}
					\begin{packed_item}
					
						\item \textbf{Glavni sudionik: }$<$Osoba zadužena za stvaranje projekata$>$
						\item  \textbf{Cilj:} $<$Dodati organizatora na projekt$>$
						\item  \textbf{Sudionici:} $<$Baza podataka$>$
						\item  \textbf{Preduvjet:} $<$Korisnik je prijavljen$>$
						\item  \textbf{Opis osnovnog tijeka:}
					
						\item[] \begin{packed_enum}
					
							\item $<$Korisnik odabire opciju "Projekti"$>$
							\item $<$Aplikacija prikazuje listu kategorija projekata$>$
							\item $<$Korisnik odabire kategoriiju projekta$>$
							\item $<$Aplikacija prikazuje listu projekata (samo imena) te kategorije$>$
							\item $<$Korisnik odabire neki od projekata$>$
							\item $<$Aplikacija prikazuje detalje o tom projektu$>$
							\item $<$Korisnik odabire opciju za dodavanje organizatora$>$
							\item $<$Aplikacija prikaže formu s emailom organizatora kojeg želi dodati$>$
							\item $<$Korisnik upiše email organizatora kojeg želi dodati$>$
							\item $<$Korisnik odabire opciju spremi$>$
							\item $<$Baza podataka se ažurira$>$
							\item $<$Aplikacija vraća korisnika na nove detalje projekta$>$
						\end{packed_enum}
					
						\item  \textbf{Opis mogućih odstupanja:}
					
						\item[] \begin{packed_item}
					
							\item[7.a] $<$Korisnik nije stvorio ovaj projekt$>$
							\item[] \begin{packed_enum}
			
								\item $<$Sustav obavještava korisnika da nije stvorio taj pojekt, pa mu ne može nadodati organizatora.
								Ovo se ne pojavljuje ako je korisnik administrator.$>$
			
							\end{packed_enum}

							\item[9.a] $<$Korisnik upiše email organizatora, ali ne odabere opciju "Spremi"$>$
							\item[] \begin{packed_enum}

								\item $<$Sustav obavještava korisnika da nije spremio podatke prije izlaska
								iz prozora i nudi mu opcije da spremi i izađe ili samo izađe$>$

							\end{packed_enum}

							\item[9.b] $<$Korisnik odabere opciju "Spremi", ali ne upiše važeći email organizatora$>$
							\item[] \begin{packed_enum}

								\item $<$Sustav obavještava korisnika da nije točan email organizatora$>$

							\end{packed_enum}

							\item[9.c] $<$Korisnik odabere opciju "Spremi", ali ne upiše email organizatora$>$
							\item[] \begin{packed_enum}

								\item $<$Sustav obavještava korisnika da mora upisati email organizatora$>$

							\end{packed_enum}

							\item[9.d] $<$Korisnik odabere opciju "Spremi", ali email koji upiše nije od organizatora$>$
							\item[] \begin{packed_enum}

								\item $<$Sustav obavještava korisnika i pita ga je li siguran da želi dodijeliti projekt
								osobi koja nije organizator$>$
								\item $<$Korisnik odabere opciju da je siguran$>$
								\item $<$Baza podataka promijeni rolu korisnika na organizatora i dodjeli mu traženi projekt$>$

							\end{packed_enum}
					
						\end{packed_item}
					\end{packed_item}

					\noindent \underbar{\textbf{UC$<$21$>$ -$<$Micanje organizatora sa projekta$>$}}
					\begin{packed_item}

						\item \textbf{Glavni sudionik: }$<$Osoba zadužena za stvaranje projekata$>$
						\item  \textbf{Cilj:} $<$Maknuti organizatora sa projekt$>$
						\item  \textbf{Sudionici:} $<$Baza podataka$>$
						\item  \textbf{Preduvjet:} $<$Korisnik je prijavljen$>$
						\item  \textbf{Opis osnovnog tijeka:}

						\item[] \begin{packed_enum}

							\item $<$Korisnik odabire opciju "Projekti"$>$
							\item $<$Aplikacija prikazuje listu kategorija projekata$>$
							\item $<$Korisnik odabire kategoriiju projekta$>$
							\item $<$Aplikacija prikazuje listu projekata (samo imena) te kategorije$>$
							\item $<$Korisnik odabire neki od projekata$>$
							\item $<$Aplikacija prikazuje detalje o tom projektu$>$
							\item $<$Korisnik odabire opciju za micanje organizatora pored organizatora$>$
							\item $<$Aplikacija prikaže popup koji pita korisnika je li siguran$>$
							\item $<$Korisnik odabere opciju da je siguran$>$
							\item $<$Baza podataka se ažurira$>$
							\item $<$Aplikacija vraća korisnika na nove detalje projekta$>$
						\end{packed_enum}

						\item  \textbf{Opis mogućih odstupanja:}

						\item[] \begin{packed_item}

							\item[7.a] $<$Korisnik nije stvorio ovaj projekt$>$
							\item[] \begin{packed_enum}

								\item $<$Sustav obavještava korisnika da nije stvorio taj pojekt, pa mu ne može maknuti organizatora.
								Ovo se ne pojavljuje ako je korisnik administrator.$>$

							\end{packed_enum}

						\end{packed_item}
					\end{packed_item}

					\noindent \underbar{\textbf{UC$<$22$>$ -$<$Dodavanje zaposlenika neprofitabilne organizacije na projekt$>$}}
					\begin{packed_item}

						\item \textbf{Glavni sudionik: }$<$Osoba zadužena za stvaranje projekata$>$
						\item  \textbf{Cilj:} $<$Dodati zaposlenika neprofitabilne organizacije na projekt$>$
						\item  \textbf{Sudionici:} $<$Baza podataka$>$
						\item  \textbf{Preduvjet:} $<$Korisnik je prijavljen$>$
						\item  \textbf{Opis osnovnog tijeka:}

						\item[] \begin{packed_enum}

							\item $<$Korisnik odabire opciju "Projekti"$>$
							\item $<$Aplikacija prikazuje listu kategorija projekata$>$
							\item $<$Korisnik odabire kategoriiju projekta$>$
							\item $<$Aplikacija prikazuje listu projekata (samo imena) te kategorije$>$
							\item $<$Korisnik odabire neki od projekata$>$
							\item $<$Aplikacija prikazuje detalje o tom projektu$>$
							\item $<$Korisnik odabire opciju za dodavanje zaposlenika neprofitabilne organizacije$>$
							\item $<$Aplikacija prikaže formu s emailom zaposlenika neprofitabilne organizacije kojeg želi dodati$>$
							\item $<$Korisnik upiše email zaposlenika neprofitabilne organizacije kojeg želi dodati$>$
							\item $<$Korisnik odabire opciju spremi$>$
							\item $<$Baza podataka se ažurira$>$
							\item $<$Aplikacija vraća korisnika na nove detalje projekta$>$
						\end{packed_enum}

						\item  \textbf{Opis mogućih odstupanja:}

						\item[] \begin{packed_item}

							\item[7.a] $<$Korisnik nije stvorio ovaj projekt$>$
							\item[] \begin{packed_enum}

								\item $<$Sustav obavještava korisnika da nije stvorio taj pojekt, pa mu ne može nadodati
								zaposlenika neprofitabilne organizacije. Ovo se ne pojavljuje ako je korisnik administrator.$>$

							\end{packed_enum}

							\item[9.a] $<$Korisnik upiše email zaposlenika neprofitabilne organizacije, ali ne odabere opciju "Spremi"$>$
							\item[] \begin{packed_enum}

								\item $<$Sustav obavještava korisnika da nije spremio podatke prije izlaska
								iz prozora i nudi mu opcije da spremi i izađe ili samo izađe$>$

							\end{packed_enum}

							\item[9.b] $<$Korisnik odabere opciju "Spremi", ali ne upiše važeći email zaposlenika neprofitabilne organizacije$>$
							\item[] \begin{packed_enum}

								\item $<$Sustav obavještava korisnika da nije točan email zaposlenika neprofitabilne organizacije$>$

							\end{packed_enum}

							\item[9.c] $<$Korisnik odabere opciju "Spremi", ali ne upiše email zaposlenika neprofitabilne organizacije$>$
							\item[] \begin{packed_enum}

								\item $<$Sustav obavještava korisnika da mora upisati email zaposlenika neprofitabilne organizacije$>$

							\end{packed_enum}

							\item[9.d] $<$Korisnik odabere opciju "Spremi", ali email koji upiše nije od zaposlenika neprofitabilne organizacije$>$
							\item[] \begin{packed_enum}

								\item $<$Sustav obavještava korisnika i pita ga je li siguran da želi dodijeliti projekt
								osobi koja nije zaposlenik neprofitabilne organizacije$>$
								\item $<$Korisnik odabere opciju da je siguran$>$
								\item $<$Baza podataka promijeni rolu korisnika na zaposlenik neprofitabilne organizacije
								i stavi ga na traženi projekt$>$

							\end{packed_enum}

						\end{packed_item}
					\end{packed_item}

					\noindent \underbar{\textbf{UC$<$23$>$ -$<$Micanje zaposlenika neprofitabilne organizacije sa projekta$>$}}
					\begin{packed_item}

						\item \textbf{Glavni sudionik: }$<$Osoba zadužena za stvaranje projekata$>$
						\item  \textbf{Cilj:} $<$Maknuti zaposlenika neprofitabilne organizacije sa projekt$>$
						\item  \textbf{Sudionici:} $<$Baza podataka$>$
						\item  \textbf{Preduvjet:} $<$Korisnik je prijavljen$>$
						\item  \textbf{Opis osnovnog tijeka:}

						\item[] \begin{packed_enum}

							\item $<$Korisnik odabire opciju "Projekti"$>$
							\item $<$Aplikacija prikazuje listu kategorija projekata$>$
							\item $<$Korisnik odabire kategoriiju projekta$>$
							\item $<$Aplikacija prikazuje listu projekata (samo imena) te kategorije$>$
							\item $<$Korisnik odabire neki od projekata$>$
							\item $<$Aplikacija prikazuje detalje o tom projektu$>$
							\item $<$Korisnik odabire opciju za micanje zaposlenika neprofitabilne organizacije pored
							zaposlenika neprofitabilne organizacije$>$
							\item $<$Aplikacija prikaže popup koji pita korisnika je li siguran$>$
							\item $<$Korisnik odabere opciju da je siguran$>$
							\item $<$Baza podataka se ažurira$>$
							\item $<$Aplikacija vraća korisnika na nove detalje projekta$>$
						\end{packed_enum}

						\item  \textbf{Opis mogućih odstupanja:}

						\item[] \begin{packed_item}

							\item[7.a] $<$Korisnik nije stvorio ovaj projekt$>$
							\item[] \begin{packed_enum}

								\item $<$Sustav obavještava korisnika da nije stvorio taj pojekt, pa mu
								ne može maknuti zaposlenika neprofitabilne organizacije.
								Ovo se ne pojavljuje ako je korisnik administrator.$>$

							\end{packed_enum}

						\end{packed_item}
					\end{packed_item}

					\noindent \underbar{\textbf{UC$<$24$>$ -$<$Pregled korisnika$>$}}
					\begin{packed_item}

						\item \textbf{Glavni sudionik: }$<$Administrator$>$
						\item  \textbf{Cilj:} $<$Pregledati korisnike$>$
						\item  \textbf{Sudionici:} $<$Baza podataka$>$
						\item  \textbf{Preduvjet:} $<$Korisnik je prijavljen$>$
						\item  \textbf{Opis osnovnog tijeka:}

						\item[] \begin{packed_enum}

							\item $<$Korisnik odabire opciju "Korisnici"$>$
							\item $<$Aplikacija prikazuje listu korisnika$>$
						\end{packed_enum}
					\end{packed_item}

					\noindent \underbar{\textbf{UC$<$25$>$ -$<$Mijenjanje role korisnika$>$}}
					\begin{packed_item}

						\item \textbf{Glavni sudionik: }$<$Administrator$>$
						\item  \textbf{Cilj:} $<$Promijeniti rolu korisnika$>$
						\item  \textbf{Sudionici:} $<$Baza podataka$>$
						\item  \textbf{Preduvjet:} $<$Korisnik je prijavljen$>$
						\item  \textbf{Opis osnovnog tijeka:}

						\item[] \begin{packed_enum}

							\item $<$Korisnik odabire opciju "Korisnici"$>$
							\item $<$Aplikacija prikazuje listu korisnika$>$
							\item $<$Korisnik odabere opciju "Promijeni rolu" koja se nalazi pored jednog od korisnika$>$
							\item $<$Aplikacija prikaže formu s mogućnosti odabira role$>$
							\item $<$Korisnik odabire jednu od rola$>$
							\item $<$Korisnik odabire opciju spremi$>$
							\item $<$Baza podataka se ažurira$>$
							\item $<$Aplikacija vraća korisnika na novu listu korisnika$>$

						\end{packed_enum}

						\item  \textbf{Opis mogućih odstupanja:}

						\item[] \begin{packed_item}

							\item[3.a] $<$Korisnik kojem se mijenja rola je administrator i ne postoji nijedan
							drugi administrator$>$
							\item[] \begin{packed_enum}

								\item $<$Sustav obavještava korisnika da mora postojati
								barem jedan administrator, pa je ova akcija zabranjena$>$

							\end{packed_enum}

						\end{packed_item}
					\end{packed_item}

					\noindent \underbar{\textbf{UC$<$26$>$ -$<$Micanje korisnika iz sustava$>$}}
					\begin{packed_item}

						\item \textbf{Glavni sudionik: }$<$Administrator$>$
						\item  \textbf{Cilj:} $<$Maknuti korisnika iz sustava$>$
						\item  \textbf{Sudionici:} $<$Baza podataka$>$
						\item  \textbf{Preduvjet:} $<$Korisnik je prijavljen$>$
						\item  \textbf{Opis osnovnog tijeka:}

						\item[] \begin{packed_enum}

							\item $<$Korisnik odabire opciju "Korisnici"$>$
							\item $<$Aplikacija prikazuje listu korisnika$>$
							\item $<$Korisnik odabere opciju micanja korisnika koja se nalazi pored jednog od korisnika$>$
							\item $<$Aplikacija pita korisnika je li siguran da želi provesti ovu akciju$>$
							\item $<$Korisnik odabire opciju da je siguran$>$
							\item $<$Baza podataka se ažurira$>$
							\item $<$Aplikacija vraća korisnika na novu listu korisnika$>$

						\end{packed_enum}

						\item  \textbf{Opis mogućih odstupanja:}

						\item[] \begin{packed_item}

							\item[3.a] $<$Korisnik kojem se miče iz sustava je administrator i ne postoji nijedan
							drugi administrator$>$
							\item[] \begin{packed_enum}

								\item $<$Sustav obavještava korisnika da mora postojati
								barem jedan administrator, pa je ova akcija zabranjena$>$

							\end{packed_enum}

						\end{packed_item}
					\end{packed_item}
					
				\subsubsection{Dijagrami obrazaca uporabe}
					
					\textit{Prikazati odnos aktora i obrazaca uporabe odgovarajućim UML dijagramom. Nije nužno nacrtati sve na jednom dijagramu. Modelirati po razinama apstrakcije i skupovima srodnih funkcionalnosti.}
				\eject		
				
			\subsection{Sekvencijski dijagrami}
				
				\textbf{\textit{dio 1. revizije}}\\
				
				\textit{Nacrtati sekvencijske dijagrame koji modeliraju najvažnije dijelove sustava (max. 4 dijagrama). Ukoliko postoji nedoumica oko odabira, razjasniti s asistentom. Uz svaki dijagram napisati detaljni opis dijagrama.}
				\eject
	
		\section{Ostali zahtjevi}
		
			\textbf{\textit{dio 1. revizije}}\\
		 
			 \textit{Nefunkcionalni zahtjevi i zahtjevi domene primjene dopunjuju funkcionalne zahtjeve. Oni opisuju \textbf{kako se sustav treba ponašati} i koja \textbf{ograničenja} treba poštivati (performanse, korisničko iskustvo, pouzdanost, standardi kvalitete, sigurnost...). Primjeri takvih zahtjeva u Vašem projektu mogu biti: podržani jezici korisničkog sučelja, vrijeme odziva, najveći mogući podržani broj korisnika, podržane web/mobilne platforme, razina zaštite (protokoli komunikacije, kriptiranje...)... Svaki takav zahtjev potrebno je navesti u jednoj ili dvije rečenice.}
			 
			 
			 
	